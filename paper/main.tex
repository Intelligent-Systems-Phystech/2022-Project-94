\documentclass{article}
\usepackage{arxiv}

\usepackage[utf8]{inputenc}
\usepackage[english, russian]{babel}
\usepackage[T1]{fontenc}
\usepackage{url}
\usepackage{booktabs}
\usepackage{amsfonts}
\usepackage{nicefrac}
\usepackage{microtype}
\usepackage{lipsum}
\usepackage{graphicx}
\usepackage{natbib}
\usepackage{doi}



\title{Hail risk prediction via Graph Neural Networks}

\author{ Ivan Lukyanenko \\
	Department of Control and Applied Mathematics\\
	Moscow Institute of Physics and Technologies\\
	Moscow \\
	\texttt{lukianenko.ia@phystech.edu} \\
	%% examples of more authors
	\And
	Yuriy Maximov \\
	Research Center in Artificial Intelligence\\
	Skoltech\\
	%% \AND
	%% Coauthor \\
	%% Affiliation \\
	%% Address \\
	%% \texttt{email} \\
	%% \And
	%% Coauthor \\
	%% Affiliation \\
	%% Address \\
	%% \texttt{email} \\
	%% \And
	%% Coauthor \\
	%% Affiliation \\
	%% Address \\
	%% \texttt{email} \\
}
\date{}

\renewcommand{\shorttitle}{Hail risk prediction}

%%% Add PDF metadata to help others organize their library
%%% Once the PDF is generated, you can check the metadata with
%%% $ pdfinfo template.pdf
\hypersetup{
pdftitle={Hail risk prediction via Graph Neural Networks},
pdfauthor={Ivan Lukyanenko},
pdfkeywords={Hail risk prediction, GNN},
}

\begin{document}
\maketitle

\begin{abstract}
	Geo-spatial time series prediction is an open area with great potential for theoretical and practical work. In particular, hail risk assessment is necessary to predict the probability of damage (agriculture, animal husbandry). The aim of our study is to build a hail forecasting model based on graph neural networks. Forecasting has been carried out in the short-term range based on the values ​​of climate variables since 1991. The key features of the problem are: 1) rare events - over the past 30 years there have been less than 700 hail events throughout Russia, 2) the spatial structure of the series data. We are expecting to improve quality of solving such problems by combining methods from~\cite{DBLP:journals/corr/abs-2012-01598} and~\cite{DBLP:journals/corr/abs-2005-07427}.
\end{abstract}


\keywords{Hail risk prediction \and GNN }

\bibliographystyle{abbrv}

\bibliography{references}

\end{document}
